\chapter{Lógica de Primeira Ordem}
    
    Até agora, vimos Lógica Proposicional, em que variáveis proposicionais podem ser valoradas como verdadeiras ou falsas. Esse sistema lógico, porém, contém limitações. Considere, por exemplo, a afirmação de que todo natural é maior do que ou igual a zero.
    Ou a afirmação de que existe número natural que é primo e é par. Como representar isso em Lógica Propocional? Precisamos de um sistema lógico que saiba lidar com objetos e suas propriedades e relações.

    Talvez esses exemplos não tenham demonstrado a necessidade de tal sistema lógico. Porém, acredite: ao final deste capítulo, essa necessidade ficará evidente.

    \section{Sintaxe}

    A sintaxe de um sistema lógico aborda basicamente os símbolos que são utilizados para representá-lo. Portanto, nesta seção, serão abordados funções, predicados, relações e quantificadores, dentre eles $\forall$ e $\exists$.

    \subsection{Funções, Predicados e Relações}

        Dentro de Lógica de Primeira Ordem, funções, predicados e relações são mapeamentos que, dado algum elemento do domínio, retornam uma proposição ou outro elemento do domínio. Parece confuso? Vamos olhar um exemplo.

        \begin{center}
            $x$ natural é par ou ímpar.            
        \end{center}

        Nosso domínio, neste caso, são os números naturais ($\mathbb{N}$). Dizemos que \textit{par} é "algo" que recebe um número e retorna $V$ ou $F$. Chamamos isso de \textbf{predicado}. Sendo assim, podemos escrever:

        \begin{center}
            $par(x) \lor impar(x)$.
        \end{center}

        A partir deste exemplo, podemos extrair alguns símbolos para exemplificar a construção de um sistema lógico de primeira ordem.

        \begin{itemize}
            \item O domínio são os naturais;
            \item Os objetos são os números $0$, $1$, $2$ etc.;
            \item Existem \textbf{funções}, como \textit{adição} e \textit{subtração}, que recebem (zero ou mais) números e retornam outros números;
            \item Existem \textbf{predicados}, como \textit{par} e \textit{ímpar}, que recebem um número e retornam $V$ ou $F$;
            \item Existem \textbf{relações}, como \textit{igual} e \textit{menor}, que recebem dois números e retornam $V$ ou $F$.
        \end{itemize}

        Os objetos pertencentes ao domínio, chamados constantes, como o $1$ e o $4$, no exemplo, podem ser considerados funções que tomam zero elementos. Além disso, podemos considerar predicados que tomam zero elementos como os valores lógicos $\top$ e $\bot$.

        Expressões que representam elementos do domínio(incluindo funções de elementos) são chamados de \textbf{termos}. Alguns exemplos:

        \begin{itemize}
            \item $5$
            \item $sucessor(10)$ (função sucessora, retorna o elemento acrescido de $1$)
            \item $33+44$
        \end{itemize}

        Observe que o símbolo para a função de adição está "infixo"; poderíamos ter representado como $+(33, 44)$ ou $adicao(33, 44)$.
        
        Expressões que retornam $V$ ou $F$ são chamadas \textbf{fórmulas}:

        \begin{itemize}
            \item $maiorDoQue(1,2)$ ($1 > 2$)
            \item $par(10) \lor impar(5)$
            \item $2=2$
        \end{itemize}

        O Lean é muito eficiente em expressar Lógica de Primeira Ordem. Vejamos nosso exemplo:

        \begin{lstlisting}
    constant U : Type

    constant zero : U
    constant par : U → Prop
    constant primo : U → Prop
    constant igual : U → U → Prop
    constant adicao : U → U → U
        \end{lstlisting}

        Pelo fato de que o Lean é baseado em \textit{Teoria dos Tipos}, declaramos um novo tipo \lstinline{U}. Podemos intuitivamente considerá-lo como "universo" ou "domínio". Por exemplo, o conjunto dos naturais.

        Foi declarado um objeto chamado \lstinline{zero} do tipo \lstinline{U} (nossa analogia com o zero natural).
        \lstinline{par} é um predicado, pois toma um elemento do tipo \lstinline{U} e retona um elemento do tipo proposição (\lstinline{Prop}).

        \lstinline{adicao} é uma função que toma dois elementos do tipo \lstinline{U} e retorna outro do mesmo tipo. Ora, podemos constatar:

        \begin{lstlisting}
    #check par zero
    #check adicao zero zero
    #check par (adicao zero zero)
        \end{lstlisting}

        \lstinline{#check} informa que a primeira expressão tem tipo \lstinline{Prop}, a segunda tem tipo \lstinline{U} e a terceira, \lstinline{Prop}. Importante observar o papel dos parênteses acima, para que \lstinline{par} receba apenas um elemento.

        Uma função (ou relação) que recebe mais de um elemento tem notação \lstinline{U → U → U}. A notação para predicados (\lstinline{U → Prop}) e relações (\lstinline{U → U → Prop}) funciona como se ambas fossem funções, porém retornassem \lstinline{Prop}. 

        Vários conjuntos estão nas bibliotecas padrão do Lean, como os naturais, utilizados nos exemplos anteriores. O comando para o símbolo \lstinline{ℕ} é \lstinline{\nat} ou \lstinline{\N}.

        \begin{lstlisting}
    constant zero : ℕ
    #check zero + zero
        \end{lstlisting}

        O \lstinline{#check} acima retorna algo do tipo \lstinline{ℕ}.

        Podemos misturar Lógica Proposicional com Lógica de Primeira Ordem:

        \begin{lstlisting}
    constant U : Type

    constant zero : U
    constant par : U → Prop
    constant primo : U → Prop
    constant igual : U → U → Prop
    constant adicao : U → U → U

    #check ¬ (par zero ∨ par (adicao zero zero)) ∧ primo zero
        \end{lstlisting}

        E o \lstinline{#check} nos retorna:
        
        \begin{lstlisting}
    ¬(par zero ∨ par (adicao zero zero)) ∧ primo zero : Prop
        \end{lstlisting}

    \subsection{Quantificador Universal}

        Grande parte do poder da Lógica Proposicional se deve aos quantificadores.
        O símbolo $\forall$ é o quantificador universal, que representa "para todo".
        Quando ele é seguido de uma variável e de uma expressão, ele indica que aquela expressão é verdadeira para toda variável do domínio.
        Por exemplo:

        \begin{itemize}
            \item $\forall x, (par(x) \lor impar(x))$
            \item $\forall y, (par(y) \rightarrow impar(y + 1))$
        \end{itemize}

        A primeira expressão nos diz que todo número é par ou ímpar (no caso dos naturais).
        A segunda diz que, para todo número, o fato dele ser par implica que seu sucessor é ímpar.
        Vejamos essas expressões no Lean:

        \begin{lstlisting}
    constant U : Type

    constant par : U → Prop
    constant impar : U → Prop
    constant sucessor : U → U

    #check ∀ x : U, par x ∨ impar x
    #check ∀ y : U, par y → impar (sucessor y)
        \end{lstlisting}

        Os dois \lstinline{#check}'s nos retornam \lstinline{Prop}.

        Observe as três sentenças:

        \begin{itemize}
            \item $\forall x, (par(x) \lor impar(x))$
            \item $\forall x, par(x) \lor impar(x)$
            \item $\forall x, (par(x)) \lor impar(x)$
        \end{itemize}

        Por uma questão de convenção, as duas últimas sentenças são equivalentes, enquanto a primeira é diferente.
        Neste contexto, estamos lidando com o \textbf{escopo} da variável $x$. A convenção diz, portanto, que o escopo da variável é o menor possível.
        
        Curiosamente, o modo como o Lean lida com escopo é diferente: \lstinline{∀ x : U, par x ∨ impar x} equivale a \lstinline{∀ x : U, (par x ∨ impar x)}.
        Ou seja, o Lean busca o maior escopo possível.

        Quando estamos lidando com quantificadores, a variável que o acompanha é dita \textbf{limitada} (\textit{bound}, em inglês).
        na expressão $\forall x, A(x)$, a variável $x$ é limitada.
        Isso significa que o $x$ não representa um valor em si, mas apenas um ``espaço reservado"\space para qualquer outra variável.
        Observe que a expressão $\forall y, A(y)$ representa exatamente a mesma coisa.

    \subsection{Quantificador Existencial}
    \section{Semântica}
    
    A seção anterior abordou a sintaxe da lógica de primeira ordem, ou seja, apresentou os símbolos e estruturas das fórmulas deste sistema. A semântica, por outro lado, define o valor-verdade das fórmulas pela \textbf{interpretação} e \textbf{valoração}.
   
   Analisemos alguns exemplos deste sistema lógico.
   Sejam:
   \begin{itemize}
       \item $\mathbb{N}$ um domínio escolhido;
       \item $0, 1, 2, 3$ símbolos constantes;
       \item \textit{adicao} e \textit{sucessor} os símbolos de função que operam neste domínio;
       \item  \textit{par}, \textit{impar}, \textit{menorDoQue},\textit{menorIgualDoQue} os simbolos de predicado .
   \end{itemize}
    A expressão 
   \begin{center}
       $\forall x \ ( \ menorDoQue(0, x) \ )$
   \end{center}
   
   é falsa, uma vez que para $x$ assumindo o valor de $0$, a relação \textit{menorDoQue} não é válida. Já a expressão
   
   \begin{center}
       $\forall x \ ( \ menorIgualDoQue( 0, x) \ )$
   \end{center}
    
    é verdadeira no domínio $\mathbb{N}$, porém se o domínio de interesse fosse $\mathbb{Z}$ a expressão se tornaria falsa.
    
    Logo, o valor verdade de uma sentença depende de como os quantificadores, o domínio, funções, predicados e relações são interpretados. 
    Porém, há algumas fórmulas que assumem sempre valor verdadeiro independente de qual for a interpretação, análogo à tautologia da lógica proposicional:
    
    \begin{center}
        $\forall x \ ( \  par(x) \rightarrow par(x) \ )$
    \end{center}
    
    Sentenças assim são chamadas \textbf{válidas}.
    
    Analogamente à valoração em lógica proposicional, há o \textbf{modelo} em lógica de primeira ordem. Enquanto a valoração permitia que atribuíssemos valores-verdade ( V ou F ) à todas as fórmulas da lógica proposicional, a escolha de um modelo permite a atribuição de valores-verdade a todas as sentenças da lógica de primeira ordem.
    
    \subsection{Interpretações}
    
    Usamos anteriormente alguns símbolos para representar predicados e constantes. Alguns deles:
    
    \begin{center}
        $0$, $1$, $ par$, $maiorDoQue$
    \end{center}
    
    
    Estes símbolos são autodescritivos considerando o domínio $\mathbb{N}$, e se torna natural sua valoração. Agora, sejam os seguintes predicados no mesmo domínio:
    
    \begin{center}
        $ligeiro$, $contente$, $facil$
    \end{center}
    
    Quando o predicado $ligeiro$ é verdadeiro? \\
    Não conseguimos responder, uma vez que não foram dadas informações suficientes.
    Se $ligeiro$ são os números ímpares, então $2, 4, 6, ...$ são  $ligeiro$ e   $1, 3, 5, ...$ não são. Seja $contente$ os múltiplos de $3$. Então $6$ é $ligeiro$ e $contente$. Se $ligeiro$ fossem os números da sequência de Fibonacci, então $6$ seria $contente$, mas não seria $ligeiro$.
    Cada explicação dada ( ímpar, múltiplos de $3$, números da sequência de Fibonacci) são \textbf{interpretações}. E vemos que é necessário a interpretação para a valoração. 
    
    
    Assim como os predicados, podemos interpretar as funções, relações e constantes:
    
    \begin{itemize}
        \item A interpretação de um predicado unário $P$ é um conjunto de elementos do domínio os quais $P$ é verdadeiro. 
        \item Para uma relação $R$ com aridade $n$, a interpretação é o conjunto de todas as tuplas com $n$ elementos para as quais $R$ é verdadeiro.
        \item E por fim, a interpretaçãode de uma função $f$ com aridade $n$, é uma função que relaciona $n$ elementos do domínio a outro elemento também do domínio.\\
     \end{itemize}
     
     É importante ressaltar a diferença entre símbolo sintático e semântica do predicado, função, relação e constante. Veja que não faz sentido escrevermos a relação $sucessorDe(4,3)$, pois $sucessorDe$ é um símbolo sintático sem significado por si só. \\
     Outra distinção importante é entre os objetos dos domínio e os símbolos constantes. Se considerarmos o domínio U de todas as cores, conhecemos os objetos deste domínio, mas podemos escrever o símbolo constante $verde$ e podemos interpretá-lo com verde ou como rosa, objetos do domínio.
     De maneira análoga, podemos definir os símbolos $0$, $1$, $2$ e interpretálos como os objetos do domínio $0$, $1$ e $2$.
     
    \subsection{Verdade em modelos}
    
     
    Na lógica proposicional a valoração dizia quais elementos deveriam ser interpretados como falsos e quais como verdadeiros. Já na lógica de primeira ordem serão avaliados cada termo e em seguida a interpretação é aplicada na estrutura. Mais adiante veremos exemplos que deixarão a ideia mais clara.
    
    Suponhamos um domínio D, em linguagem de primeira ordem, e uma interpretação em D para cada símbolo da linguagem. Um \textbf{modelo} é esta estrutura formada por um domínio D e a interpretação relativa a este domínio. Um modelo fornece as informações necessárias para avaliarmos todas as sentenças da linguagem em verdadeiro ou falso.
    
    Vamos relembrar a diferença entre termo e sentença e descrever as respectivas interpretações:
    
    \begin{itemize}
        \item  \textit{Termos} representam objetos e não possuem valor verdade. São termos: $a + b, f(x), c$. E a sua interpretação é definida pelo próprio modelo e são elementos do domínio. Para interpretar $a + b$, verificamos a interpretação do modelo para cada termo $a$ e $b$ e em seguida aplicamos a interpretação de $+$ à estes termos. Da mesma forma, para $f(x)$, analizamos a interpretação do termo $x$, dada pelo modelo e plicamos a interpretação de $f$ ao termo $x$.
        
        \item \textit{Sentenças} são relações ou predicados que assumem valor verdade. Alguns exemplos: $ R(a,b), x + y < x, P(a)$. Para interpretar um predicado ou uma relação primeiro se interpreta os termos como objetos do domínio, e em seguida verificar se a interpretação do símbolo da relação é verdadeira para estes objetos do domínio.
        
        
    \end{itemize}
    
    Seja A uma sentença e $\mathbb{M}$ um modelo da linguagem de A. Por questão de praticidade, vamos adotar as notações $\mathbb{M} \vDash A$ (pode ser lido como \textit{modela}, \textit{satisfaz} ou \textit{valida}) para quando o modelo $\mathbb{M}$ avalia a sentença $A$ como verdadeira ($ \textbf{T}$) e $\mathbb{M} \not \vDash A$ para quando $\mathbb{M}$ avalia $A$ como falsa ($ \textbf{F}$).
       
    \subsection{Exemplos}
    
    \subsection{Validação e consequência lógica}
    
    \subsection{Correção e completude}
    
    \begin{table}[htb]
      \centering
      \begin{tabular}{|l|l|l|l|}
      \hline
      G          & G          & B     &b           \\ \hline
      b          & b          & G     &G           \\ \hline
      g          & B          & b     &b            \\ \hline
      G          & g          & G     &B            \\ \hline
      
      \end{tabular}
      \end{table}

    \section{Dedução Natural}
Como visto na seção anterior, devemos estabelecer as regras de inclusão e exclusão
de dedução para os quantificadores e para a igualdade. Listamos elas inicialmente:
\newline \textbf{Quantificador universal:}
 \newline 
 \begin{center}
 \begin{bprooftree}
    \AxiomC{$A(y)$}
    \RightLabel{\scriptsize $\forall I$}
    \UnaryInfC{$\forall x A(x)$}
\end{bprooftree}
\begin{bprooftree}
    \AxiomC{$\forall x A(x)$}
    \RightLabel{\scriptsize $\forall E$}
    \UnaryInfC{$A(t)$} 
\end{bprooftree}
\end{center}

Para inserir o quantificador universal devemos ter que a variável $y$ não deve estar 
atrelada em nenhuma hipótese, isto é, em todas as hipóteses não canceladasela não pode ser livre. 
\newline \textbf{Quantificador existencial:}

 \begin{center}
    \begin{bprooftree}
        \AxiomC{$A(t)$}
        \RightLabel{\scriptsize $\exists I$}
        \UnaryInfC{$\exists x A(x)$}
    \end{bprooftree}
    \begin{bprooftree}
        \AxiomC{$\exists x A(x)$}
        \AxiomC{}
        \UnaryInfC{$A(t)$}
        \alwaysNoLine
        \UnaryInfC{$\vdots$}
        \UnaryInfC{$B$}
        \alwaysSingleLine
        \RightLabel{\scriptsize $\exists E$}
        \BinaryInfC{$B$}
    \end{bprooftree}
 \end{center}

Para retirarmmos o quantificador existencial, a variável $t$ não pode estar livre em $B$, isto é, ela não 
deve estar livre em qualquer hipótese não cancelada.

\textbf{Igualdade:}
\begin{center}
    \begin{bprooftree}
        \AxiomC{}
        \RightLabel{\scriptsize refl}
        \UnaryInfC{$t=t$}
    \end{bprooftree}
    \begin{bprooftree}
        \AxiomC{$t=s$}
        \RightLabel{\scriptsize sim}
        \UnaryInfC{$s=t$} 
    \end{bprooftree}
\end{center}
\begin{center}
    \begin{bprooftree}
        \AxiomC{$t=s$}
        \AxiomC{$s=v$}
        \RightLabel{\scriptsize trans}
        \BinaryInfC{$t=v$}  
    \end{bprooftree}
    \begin{bprooftree}
        \AxiomC{$t=s$}
        \RightLabel{\scriptsize subs}
        \UnaryInfC{$r(t) = r(s)$}
    \end{bprooftree}
    \begin{bprooftree}
        \AxiomC{$t=s$}
        \AxiomC{$P(t)$}
        \RightLabel{\scriptsize subs}
        \BinaryInfC{$P(s)$}
    \end{bprooftree}
\end{center}
Iremos passar por cada um destas regras para apresentá-las com o auxílio de exemplos.

\subsection{Quantificador universal}
Como primeiro exemplo segue abaixo uma dedução natural de 
$(\forall x P(x) \to \forall x Q(x)) \to \forall x (P ( x) \to Q (x))$. 
Note que apesar de parecer uma implicação de duas fórmulas idênticas, na premissa a propriedade
 $P$ recebe uma variável e a propriedade $Q$ pode receber outra variável, enquanto que na conclusão
  o valor que $x$ assume é o mesmo para $P$ e $Q$.

\begin{center}
    \AxiomC{}
    \RightLabel{\scriptsize 1}
    \UnaryInfC{$\forall x P(x) \to \forall x Q(x)$}
    \RightLabel{\scriptsize $\forall E$}
    \UnaryInfC{$P(t) \to \forall x Q(x)$}
    \RightLabel{\scriptsize $\forall E$}
    \UnaryInfC{$P(t) \to Q(t)$}
    \RightLabel{\scriptsize $\forall I$}
    \UnaryInfC{$\forall x (P(x) \to Q(x))$}
    \RightLabel{\scriptsize 1}
    \UnaryInfC{$\forall x P(x) \to \forall x Q(x) \to \forall x (P(x) \to Q(x))$}
    \DisplayProof
\end{center}

O primeiro passo será a exclusão da implicação (a implicação principal da fórmula), 
assumindo $\forall x P(x) \to \forall x Q(x)$ como nossa hipótese. Aplicamos uma primeira exclusão do
 universal em $\forall x P(x)$ e novamente aplicamos a exclusão do universal em $\forall x Q(x)$. Note que 
 ao definirmos a variável com o mesmo nome $t$ em ambas as exclusões do universal foi o que permitiu inserir 
 um universal que englobe ambas variáveis, que é o passa seguinte.
\newline Vamos para um outro exemplo, utilizando as hipóteses 
$\forall x (\neg Q(x) \to R(x))$ e $\forall x(P(x) \land \neg Q(x))$ provaremos $\forall x R(x)$:

\begin{center}
    \AxiomC{$\forall x(P(x) \land \neg Q(x))$}
    \RightLabel{\scriptsize $\forall E$}
    \UnaryInfC{$P(t) \land \neg Q(t)$}
    \UnaryInfC{$\neg Q(t)$}
    \AxiomC{$\forall x (\neg Q(x) \to R(x))$}
    \RightLabel{\scriptsize $\forall E$}
    \UnaryInfC{$\neg Q(t) \to R(t)$}
    \BinaryInfC{$R(t)$}
    \RightLabel{\scriptsize $\forall I$}
    \UnaryInfC{$\forall x R(x)$}
    \DisplayProof
\end{center}
Vamos inicialmente utilizar de nossas hipóteses, em $\forall x(P(x) \land \neg Q(x))$ nós executamos a primeira 
exclusão do universal, utilizando a variável $t$ e em seguida executamos a exclusão do $\land$ 
visto que não é necessário o $P(t)$ na dedução. Do outro lado executamos mais uma exclusão do universal
 em $\forall x (\neg Q(x) \to R(x))$ e atribuímos novamente a mesma variável $t$. Com $\neg Q(t)$ e $\neg Q(t) \to R(t)$ 
 podemos realizar uma exclusão da implicaçã e por fim incluímos o universal em $\forall 
 I$.

\subsection{Quantificador existencial}
Lembrando a regra de exclusão do existencial, o que fazemos é que com $\exists x A(x)$, supomos um $y$
arbitrário que satisfaça $A(y)$, a partir desta premissa chegamos até $B$, uma fórmula que não contém
$y$ ou qualquer outra variável aberta em alguma hipótese não cancelada, e podemos concluir $B$.
\newline Já a regra da inclusão do existencial, se uma propriedade vale para um $y$ arbitrário, então
existe uma váriavel em que ela é válida.
\newline Iniciemos com uma demosntração simples de que se existe $x$ que satisfaça $A$ e $B$, então 
existe $x$ que satisfaça $A$.

\begin{center}
    \AxiomC{}
    \RightLabel{\scriptsize 1}
    \UnaryInfC{$\exists x (A(x) \land B(x))$}
    \AxiomC{}
    \RightLabel{\scriptsize 2}
    \UnaryInfC{$A(t) \land B(t)$}
    \UnaryInfC{$A(t)$}
    \UnaryInfC{$\exists x A(x)$}
    \RightLabel{\scriptsize 2}
    \BinaryInfC{$\exists x A(x)$}
    \RightLabel{\scriptsize 1}
    \UnaryInfC{$\exists x (A(x) \land B(x)) \to \exists x A(x)$}
    \DisplayProof
\end{center}
A primeira etapa foi a retirada da implicação, passando $\exists x (A(x) \land B(x))$ como uma
hipótese, e utilizando ela aplicamos a regra de exclusão do existencial, criando uma variável
$t$ arbitrária em que $A(x) \land B(x)$ sejá válido, com isso podemos concluir $A(t)$, note que em $2$
não podemos concluir $A(t)$, pois a variável ainda está aberta em $A(t)\land B(t)$, por isso inserimos
o existencial e concluímos $\exists x A(x)$, resultado que queríamos obter.
\newline O próximo exemplo relaciona os quantificadores universal e existencial, se para todo $x$ $A$
é válido, então existe algum $x$ que $A$ sejá válido.
\begin{center}
    \AxiomC{$\forall x A(x)$}
    \UnaryInfC{$A(t)$}
    \UnaryInfC{$\exists x A(x)$}
    \DisplayProof
\end{center}
Note que se $A(x)$ vale para todo $x$, também vale para um $t$ específico e no passo seguinte poderíamos
tanto utilizar a regra da inclusão do universal ou do existencial. Outro comentário relevante é que não
necessariamente precisávamos concluir o existencial utilizando a mesma variável $x$.
\newline Vamos provar mais uma relação entre os quantificadores, iremos que provar que se para todo $x$ 
não vale $A$, então não existe $x$ tal que $A$ valha:
\begin{center}
    \AxiomC{}
    \RightLabel{\scriptsize 2}
    \UnaryInfC{$\exists x A(x)$}
    \AxiomC{}
    \RightLabel{\scriptsize 3}
    \UnaryInfC{$A(t)$}
    \AxiomC{}
    \RightLabel{\scriptsize 1}
    \UnaryInfC{$\forall x \neg A(x)$}
    \UnaryInfC{$ \neg A(t)$}
    \BinaryInfC{$\bot $}
    \RightLabel{\scriptsize 3}
    \BinaryInfC{$\bot$}
    \RightLabel{\scriptsize 2}
    \UnaryInfC{$\neg \exists x A(x)$}
    \RightLabel{\scriptsize 1}
    \UnaryInfC{$\forall x \neg A(x) \to \neg \exists x A(x)$}
    \DisplayProof
\end{center}
Para a nossa dedução a primeira etapa foi desmontar a implicação, passando $\forall x \neg A(x)$ como
uma de nossas hipóteses, em seguida, como temos uma negação devemos chegar até ao falso. Utilizamos 
$\exists x A(x)$ como mais uma de nossas hipóteses e aplicamos a regra de exclusão do existencial,
dessa forma obtemos no mesmo ramo $A(t)$ e $\neg A(t)$, obtendo a contradição que estávamos procurando.
\subsection{Igualdade}
TO DO 
\subsection{Exercícios}
TO DO
\subsection{Dedução natural no LEAN}
No Lean a dedução ocorre de forma similar a dedução em primeira ordem, apenas devemos utilizar de novos símbolos e das 
regras de exclusão e inclusão dos quantificadores. Os símbolos $\exists $ e $\forall $ são escritos no Lean como \textbackslash exist 
e  \textbackslash all. Para a regra de exclusão do universal apenas passamos para uma proposição $\forall x A(x)$ uma letra,
por exemplo $t$, para termos $A(t)$. Para a inclusão do universal, devemos assumir uma letra, por exemplo $t$, utilizando o
``assume" e com esta letra livre de qualquer hipótese provamos $A(t)$, dessa forma o Lean é capaz de inferir $\forall x A(x)$.
\newline Vamos utilizando o Lean provar o nosso primeiro exemplo do quantificador universal, $\forall xP(x) \to \forall x Q(x) \to \forall x(P(x) \land Q(x))$:
\begin{lstlisting}
 variable U : Type
 variables P Q : Type $\to$ Prop

 example : ($\forall$x, P x) $\to$ ($\forall$x, Q x) $\to \forall$x P x $\land$ Q x :=  
 assume h$_1$ : $\forall$x, P x,
 assume h$_2$ : $\forall$x, Q x,
 assume t,
 have h$_3$ : P t, from h$_1$ t,
 have h$_4$ : Q t, from h$_2$ t,
 show P t $\land$ Q t, from and.intro h$_3$ h$_4$ 
\end{lstlisting}
Note que na linha 4 utilizamos da inclusão do universal, assumimos um $t$ e desejamos provar $P(t) \land Q(t)$ 
sem que $t$ possua qualquer restrição e nas linhas 5 e 6 utilizamos a exclusão do universal, temos fórmulas do
tipo $\forall x P(x)$ e passamos a letra $t$, obtendo $P(t)$. Além disso note nesse exemplo a importância da utilização
dos parenteses para delimitar a ação do quantificador, quando definimos o exemplo, caso não tivessemos colocado
o parenteses em  $(\forall x, P x)$, o primeiro quantificador $\forall x$ seria interpretado como se valesse para toda
a expressão, incluindo as duas implicações.
\newline Vamos também provar o segundo exemplo em Lean:
\begin{lstlisting}
variable U : Type
variables P Q R : U $\to$ Prop

example (h$_1$ : $\forall$ x, P x $\land \neg$ Q x) (h$_2$ : $\forall$ x, $\neg$ Q x $\to$ R x) :
$\forall$ x, R x :=
assume t,
have h$_3$ : P t $\land$ $\neg$ Q t, from h$_1$ t,
have h$_4$ : $\neg$ Q t $\to$ R t, from h$_2$ t,
show R t, from h$_4$ h$_3$.right
\end{lstlisting}
Note que neste simples exemplo, apesar de possuirmos fórmulas maiores, apenas realizamos simples regras de inclusão e exclusão
do universal.
\newline Para as regras do existencial, na exclusão utilizamos ``exists.elim" seguida por uma proposição do tipo $\forall x A(x)$,
para provarmos $B$, devemos provar $ A y \to B$, ou seja, assumimos um $y$ e $A y$ e chegamos até $B$, concluindo assim a exclusão
do existencial. Para a inclusão do existencial utilizamos ``exists.intro", devemos passar uma letra, por exemplo $t$, e uma prova
de que $A(t)$ vale.
\newline Vamos provar o nosso primeiro exemplo existencial com o Lean, $\exists x (A(x) \land B(x)) \to \exists x A(x)$:
\begin{lstlisting}
variable U : Type
variables A B : U $\to$ Prop

example : ($\exists$x, A x $\land$ B x) $\to \exists$x, A x :=
assume h$_1$ : $\exists$ x, A x $\land$ B x,
exists.elim h$_1$
    (assume t (h$_2$ : A t $\land$ B t)
        show $\exists$ x, A x, from exists.intro t h$_2$.left) 
\end{lstlisting}
\subsection{Exercícios}
\textbf{1.}Resolve as seguintes deduções naturais:
\newline \textbf{a)} $\forall x A(x) \to \neg \exists x \neg A(x)$.
\newline \textbf{SOLUÇÃO:}

\begin{lstlisting}
variable U : Type
variables A : U → Prop
    
    
--term mode
example : (∀ x, A x) → (¬ ∃ x, ¬ A x) :=
assume h1 : ∀ x, A x,
assume h2 : ∃ x, ¬ A x, show false, from
    (exists.elim h2
        (assume t (h3 : ¬ P t), 
            have h4 : A t, from h1 t, h3 h4))
    
--tatics
example : (∀ x, A x) → (¬ ∃ x, ¬ A x) :=
begin
intros h1,
intro h2,
apply exists.elim h2,
intro t,
intro h3,
exact h3 (h1 t)
end 
\end{lstlisting}
\textbf{b)} $\exists x \neg A (x) \to \neg \forall x A(x)$.
\newline \textbf{SOLUÇÃO:} 
\begin{lstlisting}
variable U : Type
variable A : U → Prop

--term mode
example : (∃ x, ¬ A x) → ¬ ∀ x, A x :=
assume h1 : ∃ x, ¬ A x,
assume h2 : ∀ x, A x, 
show false, from 
    (exists.elim h1
        (assume t (h3 : ¬ A t),
            have h4 : A t, from h2 t,
            h3 h4))
        
--tatics

example : (∃ x, ¬ A x) → ¬ ∀ x, A x :=
begin
intro h1,
intro h2,
apply exists.elim h1,
intro t,
intro h3,
exact h3 (h2 t)
end
\end{lstlisting}
\textbf{c)}
\textbf{d)}
\textbf{e)}
\newline \textbf{2.} Retornando ao capítulo anterior com o exercício de Ana, Cláudia e Maria e seus três vestidos
agora podemos resolver o problema utilizando de proposição de primeira ordem. Na situação, temos: 
\newline Três irmãs - Ana, Maria e Cláudia - foram a uma festa com vestidos de
cores diferentes. Uma vestia azul, a outra branco e a Terceira
preto. Chegando à festa, o anfitrião perguntou quem era cada uma
delas. As respostas foram:
\newline - A de azul respondeu: “Ana é a que está de branco”
\newline - A de branco falou: “Eu sou Maria”
\newline - A de preto disse:  “Cláudia é quem está de branco”
\newline O anfitrião foi capaz de identificar corretamente quem era cada pessoa
considerando que:
\newline - Ana sempre diz a verdade
\newline - Maria às vezes diz a verdade
\newline - Cláudia nunca diz a verdade
\newline Pensando um pouco sobre o problema, pode-se concluir que a Ana estava
com o vestido preto, a Cláudia com o branco e a Maria com o
azul.  
\newline \textbf{a)} Escreva as fórmulas de primeira ordem necessárias para o problema.
\newline \textbf{SOLUÇÃO:}
\begin{lstlisting}
section 
variable {pessoa : Type} 
variable {cor : Type}
variables {Ana Maria Claudia : pessoa}
variables {preto branco azul : cor}
variable {veste : pessoa → cor → Prop}

-- Restricoes:

-- Se uma usa uma cor, as outras nao podem usar essa cor
variable r1 : ∀c, ∀x, ∀y, (veste x c) ∧ ¬(x = y) → ¬(veste y c)
-- Se uma pessoa usa tal cor, nao pode usar as outras cores
variable r2 : ∀x, ∀c, ∀d, (veste x c) ∧ ¬(c = d) → ¬(veste x d)

-- Raciocinios basicos:

--Se Ana for a de azul, ela esta de branco
variable h1 : (veste Ana azul) → (veste Ana branco)
--Se Ana for a de branco, entao Maria esta de branco
variable h2 : (veste Ana branco) → (veste Maria branco)
--Se ana for de preto, entao Claudia esta de branco
variable h3 : (veste Ana preto) → (veste Claudia branco)
--Se Ana não usa azul e nem branco, ela esta de preto
variable h4 : ¬ (veste Ana azul) ∧ ¬ (veste Ana branco)  →  (veste Ana preto)
--Se Ana veste preto e Claudia veste branco, entao Maria esta de azul
variable h5 : (veste Ana preto) ∧ (veste Claudia branco) →  (veste Maria azul)

variable d1 : ¬ (azul = branco)
variable d2 : ¬ (Ana = Maria)
-- prova de que Ana veste preto
lemma Ana_p : (veste Ana preto) :=
have n1 : ¬ (veste Ana azul), from 
assume m1 : veste Ana azul, show false, from 
    have q1 : veste Ana branco, from h1 m1,
    have q2 : ¬ veste Ana branco, from (r2 Ana azul branco (
        and.intro m1 d1)),
q2 q1,
have n2 : ¬ (veste Ana branco), from
assume m2 : veste Ana branco, show false, from
    have q1 : veste Maria branco, from h2 m2,
    have q2 : ¬ veste Maria branco, from (r1 branco Ana Maria (and.intro m2 d2)),
q2 q1,
show veste Ana preto, from h4 (and.intro n1 n2)

theorem solution : (veste Ana preto) ∧ (veste Claudia branco) ∧ (veste Maria azul) :=
have n1 : veste Ana preto, from (Ana_p r1 r2 h1 h2 h4 d1 d2),
have n2 : veste Claudia branco, from h3 (Ana_p r1 r2 h1 h2 h4 d1 d2),
have n3 : veste Maria azul, from h5 (and.intro n1 n2),
and.intro n1 (and.intro n2 n3)

end
\end{lstlisting}
\textbf{SOLUÇÃO COM TÁTICAS}:
\begin{lstlisting}
    
\end{lstlisting}
